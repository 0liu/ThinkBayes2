\sloppy
%\setlength{\topmargin}{-0.375in}
%\setlength{\oddsidemargin}{0.0in}
%\setlength{\evensidemargin}{0.0in}

% Uncomment these to center on 8.5 x 11
%\setlength{\topmargin}{0.625in}
%\setlength{\oddsidemargin}{0.875in}
%\setlength{\evensidemargin}{0.875in}

%\setlength{\textheight}{7.2in}

\setlength{\headsep}{3ex}
\setlength{\parindent}{0.0in}
\setlength{\parskip}{1.7ex plus 0.5ex minus 0.5ex}
\renewcommand{\baselinestretch}{1.02}

% see LaTeX Companion page 62
\setlength{\topsep}{-0.0\parskip}
\setlength{\partopsep}{-0.5\parskip}
\setlength{\itemindent}{0.0in}
\setlength{\listparindent}{0.0in}

% see LaTeX Companion page 26
% these are copied from 
% /usr/local/teTeX/share/texmf/tex/latex/base/book.cls
% all I changed is afterskip

\makeatletter

\renewcommand{\section}{\@startsection 
    {section} {1} {0mm}%
    {-3.5ex \@plus -1ex \@minus -.2ex}%
    {0.7ex \@plus.2ex}%
    {\normalfont\Large\bfseries}}
\renewcommand\subsection{\@startsection {subsection}{2}{0mm}%
    {-3.25ex\@plus -1ex \@minus -.2ex}%
    {0.3ex \@plus .2ex}%
    {\normalfont\large\bfseries}}
\renewcommand\subsubsection{\@startsection {subsubsection}{3}{0mm}%
    {-3.25ex\@plus -1ex \@minus -.2ex}%
    {0.3ex \@plus .2ex}%
    {\normalfont\normalsize\bfseries}}

% The following line adds a little extra space to the column
% in which the Section numbers appear in the table of contents
\renewcommand{\l@section}{\@dottedtocline{1}{1.5em}{3.0em}}
\setcounter{tocdepth}{1}

\makeatother

\newcommand{\adjustpage}[1]{\enlargethispage{#1\baselineskip}}


% Note: the following command seems to cause problems for Acroreader
% on Windows, so for now I am overriding it.
%\newcommand{\clearemptydoublepage}{
%            \newpage{\pagestyle{empty}\cleardoublepage}}
\newcommand{\clearemptydoublepage}{\cleardoublepage}

%\newcommand{\blankpage}{\pagestyle{empty}\vspace*{1in}\newpage}
\newcommand{\blankpage}{\vspace*{1in}\newpage}

% HEADERS

\renewcommand{\chaptermark}[1]{\markboth{#1}{}}
\renewcommand{\sectionmark}[1]{\markright{\thesection\ #1}{}}

\lhead[\fancyplain{}{\bfseries\thepage}]%
      {\fancyplain{}{\bfseries\rightmark}}
\rhead[\fancyplain{}{\bfseries\leftmark}]%
      {\fancyplain{}{\bfseries\thepage}}
\cfoot{}

\pagestyle{fancyplain}

% colors for code listings and output
\usepackage{xcolor}
\definecolor{bgcolor}{HTML}{FAFAFA}
\definecolor{comment}{HTML}{007C00}
\definecolor{keyword}{HTML}{0000FF}
\definecolor{strings}{HTML}{B20000}

% syntax highlighting in code listings
\usepackage{textcomp}
\usepackage{listings}
\lstset{
    language=python,
    basicstyle=\ttfamily,
    backgroundcolor=\color{bgcolor},
    commentstyle=\color{comment},
    keywordstyle=\color{keyword},
    stringstyle=\color{strings},
    columns=fullflexible,
    emph={label},  % keyword?
    keepspaces=true,
    showstringspaces=false,
    upquote=true,
    xleftmargin=15pt,  % \parindent
    framexleftmargin=3pt,
    aboveskip=\parskip,
    belowskip=\parskip,
    mathescape=true
}

% code listing environments
\lstnewenvironment{code}
{\minipage{\linewidth}}
{\endminipage}

\lstnewenvironment{stdout}
{\lstset{commentstyle=,keywordstyle=,stringstyle=}\minipage{\linewidth}}
{\endminipage}

% inline syntax formatting
\newcommand{\py}[1]{\lstinline{#1}}%{

% the following is a brute-force way to prevent the headers
% from getting transformed into all-caps
\renewcommand\MakeUppercase{}

% Exercise environment
\newtheoremstyle{myex}% name
     {9pt}%      Space above
     {9pt}%      Space below
     {}%         Body font
     {}%         Indent amount (empty = no indent, \parindent = para indent)
     {\bfseries}% Thm head font
     {}%        Punctuation after thm head
     {0.5em}%     Space after thm head: " " = normal interword space;
           %       \newline = linebreak
     {}%         Thm head spec (can be left empty, meaning `normal')

\theoremstyle{myex}
